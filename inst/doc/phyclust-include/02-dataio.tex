\section[Sequence Data Input and Output]{Sequence Data Input and Output}
\label{sec:dataio}
\addcontentsline{toc}{section}{\thesection. Sequence Data Input and Output}

Two type of sequences are supported in the \pkg{phyclust},
nucleotide and SNP, and the types are stored in \code{.code.type} as
\begin{Code}
> .code.type
[1] "NUCLEOTIDE" "SNP" 
\end{Code}
There are three input sources of sequence data in the \pkg{phyclust}:
\begin{enumerate}
\item
Read the data from a text file in the PHYLIP format (Section~\ref{sec:phylip}).
\item
Read the data from a test file in the FASTA format (Section~\ref{sec:fasta}).
\item
Simulated by \code{ms()} and \code{seqgen()} (Section~\ref{sec:msseqgen}).
\end{enumerate}
The reading functions \code{read.*()} will return a list object with
class {\color{red} \code{seq.data}} (Section~\ref{sec:phylip}),
and transfered data based on the standard coding and
stored in a major element {\color{red} \code{org}}
exactly used in most functions of the \pkg{phyclust}.
There are two ways to output sequence data in the \pkg{phyclust},
either the PHYLIP format or the FASTA format.





\subsection[Standard coding]{Standard coding}
\label{sec:coding}
\addcontentsline{toc}{subsection}{\thesubsection. Standard coding}

I use several internal objects to store default ids, and two of them
are related to sequence structure, \code{.nucleotide} and \code{.snp},
which store the mapping information in \code{data.frame}.
They are used to transfer data when reading and writing sequences.
By typing the names, we can see the details as
\begin{Code}
> .nucleotide
  nid code code.l
1   0    A      a
2   1    G      g
3   2    C      c
4   3    T      t
5   4    -      -
> .snp
  sid code
1   0    1
2   1    2
3   2    -
\end{Code}
The headings \code{nid} and \code{sid} are standard ids used in
\pkg{phyclust}, and \code{code} and \code{code.l} are general syntax 
for nucleotide $\{\A, \G, \C, \T\}$ and SNP $\{1, 2\}$ sequences.
The standard ids will be directly passed to the kernel of \pkg{phyclust}
in \proglang{C} for efficient optimizations, so sequences are coded by
integers stated from 0.
Note that I use "\code{-}" to indicate gaps and other non general syntax.
The computations and functions to deal with "\code{-}" are developing.


\subsection[PHYLIP format]{PHYLIP format}
\label{sec:phylip}
\addcontentsline{toc}{subsection}{\thesubsection. PHYLIP format}

An example is "Great pony 524 EIAV rev dataset" \citep{Baccam2003},
and you can view the file as
\begin{Code}
> data.path <- paste(.libPaths()[1], "/phyclust/data/pony524.phy", sep = "")
> edit(file = data.path)
\end{Code}
Here is the first 5 sequences and the first 50 sites. The first line
says that there are 146 sequences and 405 sites in this files. The sequences
are started from the second line, and the first 10 characters are reserved for
the sequence's name or id.
\begin{verbatim}
 146 405
AF314258     gatcctcagg gccctctgga aagtgaccag tggtgcaggg tcctccggca
AF314259     gatcctcagg gccctctgga aagtgaccag tggtgcaggg tcctccggca
AF314260     gatcctcagg gccctctgga aagtgaccag tggtgcaggg tcctccggca
AF314261     gatcctcagg gccctctgga aagtgaccag tggtgcaggg tcctccggca
AF314262     gatcctcagg gccctctgga aagtgaccag tggtgcaggg tcctccggca
\end{verbatim}

By default, the \code{read.phylip()} will read in a PHYLIP file and
assume the file contains nucleotide sequences. It will read in sequences
and store in a list object with class \code{seq.data}, and the element
\code{org.code} store the original data in a character matrix, and
the element \code{org} store the transfered original data in a numerical
matrix. The transfered data are based on the standard coding in
the Section~\ref{sec:coding}.
The following is an example to read the pony524 dataset.
\begin{Code}
> data.path <- paste(.libPaths()[1], "/phyclust/data/pony524.phy", sep = "")
> (my.pony.524 <- read.phylip(data.path))
code.type: NUCLEOTIDE, n.seq: 146, seq.len: 405.
> str(my.pony.524)
List of 7
 $ code.type: chr "NUCLEOTIDE"
 $ info     : chr " 146 405"
 $ nseq     : num 146
 $ seqlen   : num 405
 $ seqname  : Named chr [1:146] "AF314258" "AF314259" "AF314260" "AF314261" ...
  ..- attr(*, "names")= chr [1:146] "1" "2" "3" "4" ...
 $ org.code : chr [1:146, 1:405] "g" "g" "g" "g" ...
 $ org      : num [1:146, 1:405] 1 1 1 1 1 1 1 1 1 1 ...
 - attr(*, "class")= chr "seq.data"
\end{Code}

Another example is "Crohn's disease SNP dataset" \citep{Hugot2001},
and the following is an example to read in SNP sequences by
changing \code{code.type} to SNP.
\begin{Code}
> data.path <- paste(.libPaths()[1], "/phyclust/data/crohn.phy", sep = "")
> (my.snp <- read.phylip(data.path, code.type = .code.type[2]))
code.type: SNP, n.seq: 1102, seq.len: 8.
\end{Code}




\subsection[FASTA format]{FASTA format}
\label{sec:fasta}
\addcontentsline{toc}{subsection}{\thesubsection. FASTA format}

An example is "Great pony 625 EIAV rev dataset" \citep{Baccam2003}
Here is the first one sequences and all 406 sites. It start with ">"
and followed by sequence's id and descriptions, then is followed by
couple lines containing sequence itself.
\begin{verbatim}
>AF512608 Equine infectious anemia virus isolate R93.3/E98.1 gp45 and rev
GATCCTCAGGGCCCTCTGGAAAGTGACCAGTGGTGCAGGGTCCTTCGGCAGTCACTACCT
GAAGAAAAAATTCCATCGCAAACATGCATCGCGAGAAGACACCTGGGACCAGGCCCAACA
CAACATACACCTAGCAGGCGTGACCGGTGGATCAGGGAACAAATACTACAGGCAGAAGTA
CTCCAGGAACGACTGGAATGGAGAATCAGAGGAGTACAACAGGCGGCCAAAGAGCTGGAT
GAAGTCAATCGAGGCATTTGGAGAGAGCTACATTTCCGAGAAGACCAAAAGGGAGATTTC
TCAGCCTGGGGCGGTTATCAACGAGCACAAGAACGGCACTGGGGGGAACAATCCTCACCA
AGGGTCCTTAGACCTGGAGATTCGAAGCGAAGGAGGAAACATTTAT
\end{verbatim}

By default, the \code{read.fasta()} will read in a FASTA file and
assume the file contains nucleotide sequences. As \code{read.phylip()},
it also return a list object with class \code{seq.data}.
The following is an example to read the pony524 dataset.
\begin{Code}
> data.path <- paste(.libPaths()[1], "/phyclust/data/pony625.fas", sep = "")
> (my.pony.625 <- read.fasta.nucleotide(data.path))
code.type: NUCLEOTIDE, n.seq: 62, seq.len: 406.
> str(my.pony.625)
List of 6
 $ code.type: chr "NUCLEOTIDE"
 $ nseq     : num 62
 $ seqlen   : int 406
 $ seqname  : chr [1:62] "AF512608" "AF512609" "AF512610" "AF512611" ...
 $ org.code : chr [1:62, 1:406] "G" "G" "G" "G" ...
 $ org      : num [1:62, 1:406] 1 1 1 1 1 1 1 1 1 1 ...
 - attr(*, "class")= chr "seq.data"
\end{Code}




\subsection[Save sequences]{Save sequences}
\label{sec:save}
\addcontentsline{toc}{subsection}{\thesubsection. Save sequences \vspace{-0.3cm}}

To save sequences in files, you can use functions \code{write.*()} which are
analogical to functions \code{read.*()} but input a data matrix \code{X} and
a file name \code{filename}.
The following I save two pony datasets in PHYLIP and FASTA formats to
the working directory.
\begin{Code}
> # PHYLIp
> write.phylip(my.pony.625$org, "new.625.txt")
> edit(file = "new.625.txt")
> # FASTA
> write.fasta(my.pony.524$org, "new.524.txt")
> edit(file = "new.524.txt")
\end{Code}
