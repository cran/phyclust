
\section[Introduction]{Introduction}
\label{sec:introduction}
\addcontentsline{toc}{section}{\thesection. Introduction}

This is a quick guide to the package \pkg{phyclust}~\citep{snoweye2011}
implementing model-based phylogenetic clustering aiming to cluster
large amount of aligned DNA or SNP sequences.
We will cover how to read and write sequence data,
how to use the popular programs \code{ms}~\citep{Hudson2002} and \code{seq-gen}~\citep{Rambaut1997} for generating
coalescent trees and molecular sequences from within \pkg{phyclust},
the main function \code{phyclust()} for finding population structure,
the tree finding program \code{baseml} of \code{PAML}~\citep{Yang1997,Yang2007},
and Haplo-Clustering \citep{tzeng2005}.
More information about the theory, other package functions,
and any changes in future versions can be found on our website
Phylogenetic Clustering (Phyloclustering) at
\url{http://thirteen-01.stat.iastate.edu/snoweye/phyclust/}.

Specifically, in Section~\ref{sec:dataio}, we introduce the basic data structures
of \pkg{phyclust} and the I/O functions for reading and writing 
PHYLIP and FASTA files. In Section~\ref{sec:msseqgen},
we demonstrate how to simulate molecular data using the ``ms+seqgen'' approach from within \proglang{R}.
In Section~\ref{sec:phyloclustering}, we briefly describe
the phylogenetic clustering method, its implementation in \code{phyclust()}, the visualization functions,
the auxiliary function \code{.EMControl()} for choosing the model, initialization method, optimization method,
and the EM algorithm variant, and propose a ``ms+seqgen+phyclust'' approach.
In Section~\ref{sec:paml}, we illustrate an extension of Phyloclustering, and
propose a ``phyclust+paml.baseml'' approach.
In Section~\ref{sec:haplo}, we demonstrate the function
\code{haplo.post.prob()} for Hap-Clustering.

\begin{color}{red}\bf Warning:\end{color}
This document is written to explain the major functions of \pkg{phyclust}
according to version \begin{color}{red}0.1-12\end{color}.
Every effort will be made to insure future versions are consistent with these
instructions, but new features in later versions may not be explained in this
document.



\subsection[Installation and quick start]{Installation and quick start}
\label{sec:installation}
\addcontentsline{toc}{subsection}{\thesubsection. Installation and quick start}

You can install \pkg{phyclust} directly from CRAN at \url{http://cran.r-project.org} or
by download from our website.
On most systems, the easiest installation method is to type the following command
into \proglang{R}'s terminal:
\begin{Code}
> install.packages("phyclust")
\end{Code}
When it finishes, you can use \code{library()} to load the package as
\begin{Code}
> library("phyclust")
\end{Code}
Note that \pkg{phyclust} requires the \pkg{ape} package \citep{Paradis2004}, and
\pkg{ape} also requires other packages depending on its version.
All the required packages will be checked and automatically loaded when
the \pkg{phyclust} is loading.

You can get started quickly with \pkg{phyclust} by using the \code{demo()} command
in \proglang{R}.
\begin{Code}
> demo("toy", package = "phyclust")
\end{Code}
This demo will produce the three plots shown in Figures~\ref{fig:toydots},
\ref{fig:toyhist} and \ref{fig:toynj}, and some of the results reported in
the Section~\ref{sec:emcontrol}.
The demo executes the commands duplicated below.
You can find more information about each command in the sections that follow.
\begin{Code}
### Rename the data and obtain true cluster membership from sequence names.
  X <- seq.data.toy$org
  X.class <- as.numeric(gsub(".*-(.)", "\\1", seq.data.toy$seqname))
### A dot plot, Figure 2.
  windows()
  plotdots(X, X.class)
### A histogram plot, Figure 3.
  windows()
  plothist(X, X.class)
### A Neighbor-Joining plot, Figure 4.
  ret <- phyclust.edist(X, edist.model = .edist.model[3])
  ret.tree <- nj(ret)
  windows()
  plotnj(ret.tree, X.class = X.class)
### Fit a EE, JC69 model using emEM initialization, Section 4.3
  EMC.2 <- .EMControl(init.procedure = "emEM")
  set.seed(1234)
  (ret.2 <- phyclust(X, 4, EMC = EMC.2))
  RRand(ret.2$class.id, X.class)
\end{Code}



\subsection[Getting help]{Getting help}
\label{sec:needhelp}
\addcontentsline{toc}{subsection}{\thesubsection. Getting help \vspace{-0.3cm}}

You can look for more examples on the help pages or our
website:
\url{http://thirteen-01.stat.iastate.edu/snoweye/phyclust/}.
Also, you can email the author at \email{phyclust@gmail.com}.
All comments are welcome.  Bugs will be fixed and
suggestions may be implemented in future versions of \pkg{phyclust}.

