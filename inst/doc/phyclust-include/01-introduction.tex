
\section[Introduction]{Introduction}
\label{sec:introduction}
\addcontentsline{toc}{section}{\thesection. Introduction}

Without further notifications, this document is written for major
functions based on version 0.1-2,
and it should work well and consistently for the later version. 

This is a quick guide for the \pkg{phyclust}, and I demonstrate
major functions in this document. They
includes reading and writing sequence data,
two famous programs \code{ms} and \code{seq-gen}
\citep{Hudson2002, Rambaut1997} for generating
a coalescent tree and sequences based on the tree
that both programs have been
incorporated into the \pkg{phyclust},
the main function \code{phyclust()} for finding sequence structures,
and Haplo-Clustering \citep{tzeng2005}.
More information about theory, examples for other tool functions
and new added functions can be found on our website:
Phylogenetic Clustering at
\url{http://thirteen-01.stat.iastate.edu/snoweye/phyclust/}.

In the Section~\ref{sec:dataio}, I introduce the basic data structures
of the \pkg{phyclust} and I/O functions for reading and writing basic
PHYLIP and FASTA files. In the Section~\ref{sec:msseqgen},
I redo the "ms+seqgen" approach in \proglang{R}.
In the Section~\ref{sec:phyloclustering}, I briefly describe the
Phylogenetic Clustering, visualization functions,
 the main function \code{phyclust()}, the auxiliary function
\code{.EMControl()} for models, initializations, optimizations,
and EM algorithms, and propose a "ms+seqgen+phyclust" approach.
In the Section~\ref{sec:haplo}, I display the function
\code{haplo.post.prob()} for Hap-Clustering.
In the Section~\ref{sec:more}, I discuss some important issues
which are in development or will be implemented in the next version.




\subsection[Installation and quick start]{Installation and quick start}
\label{sec:installation}
\addcontentsline{toc}{subsection}{\thesubsection. Installation and quick start}

You can install directly from CRAN at \url{http://cran.r-project.org} or
download the \pkg{phyclust} from our website.
In most systems, you can install the \pkg{phyclust} by typing the command
into the \proglang{R}'s terminal as
\begin{Code}
> install.packages("phyclust")
\end{Code}
When it finishes, you can use \code{library()} to load the package as
\begin{Code}
> library("phyclust")
\end{Code}
Note that the \pkg{phyclust} requires the
\pkg{ape} package \citep{Paradis2004}, and
the \pkg{ape} also requires other packages depending on its version.
All the required packages will be checked and automatically loaded when
the \pkg{phyclust} is loading.

You can have a quick start by using the \code{demo()} command
in \proglang{R}.
\begin{Code}
> demo("toy", package = "phyclust")
\end{Code}
This will give you three plots as the Figure~\ref{fig:toydots},
\ref{fig:toyhist} and \ref{fig:toynj}, and a partial result in
the Section~\ref{sec:emcontrol}.




\subsection[Need help]{Need help}
\label{sec:needhelp}
\addcontentsline{toc}{subsection}{\thesubsection. Need help \vspace{-0.3cm}}

You can look and check more examples from the help pages or our
website:
\url{http://thirteen-01.stat.iastate.edu/snoweye/phyclust/}.
Also, you can mail to \email{phyclust@gmail.com}.
All commands are welcome, and bugs for the
\pkg{phyclust} package or suggestions for Phylogenetic Clustering
will be fixed and implemented in the new version.

