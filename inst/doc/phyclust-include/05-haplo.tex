\section[Use the haplo.post.prob() function for Hap-Clustering]{Use the \code{haplo.post.prob()} function for Hap-Clustering}
\label{sec:haplo}
\addcontentsline{toc}{section}{\thesection. Use the \code{haplo.post.prob()} function for Hap-Clustering \vspace{-0.3cm}}

Haplotype Grouping \citep{tzeng2005} for SNP datasets
is a simplified/degenerated method of Phyloclustering.
The author's code has been integrate into the \pkg{phyclust} package,
and the original function has been renamed as \code{haplo.post.prob()}.
The example used by the author is the Crohn's disease dataset
\citep{Hugot2001} which is also built in the package.

The following example returns the same results as \cite{tzeng2005}, and
the predicted number of clusters based on the information criterion is 13.
The function returns a list object and stores in \code{ret} that
\code{ret$haplo} stores information for the SNP sequences,
\code{ret$FD.id} and \code{ret$RD.id} stores full and reduced dimensional
index, 
\code{ret$FD.post} and \code{ret$RD.post} stores full and reduced dimensional
posterior probabilities, and \code{g.truncate} show the truncated results.
\begin{Code}
> data.path <- paste(.libPaths()[1], "/phyclust/data/crohn.phy", sep = "")
> my.snp <- read.phylip.snp(data.path)
> ret <- haplo.post.prob(my.snp$org, ploidy = 1)
> str(ret)
List of 6
 $ haplo     :List of 6
  ..$ haplotype: num [1:39, 1:8] 0 1 1 0 1 1 0 1 1 0 ...
  ..$ hap.prob : num [1:39] 0.00454 0.00181 0.11797 0.00635 0.00635 ...
  ..$ post     : num [1:1102] 1 1 1 1 1 1 1 1 1 1 ...
  ..$ hap1code : int [1:1102] 1 1 1 1 1 2 2 3 3 3 ...
  ..$ hap2code : int [1:1102] 1 1 1 1 1 2 2 3 3 3 ...
  ..$ indx.subj: int [1:1102] 1 2 3 4 5 6 7 8 9 10 ...
 $ FD.id     : int [1:39] 3 9 18 22 27 28 30 31 34 35 ...
 $ RD.id     : int [1:13] 3 9 18 22 27 28 30 31 34 35 ...
 $ FD.post   : num [1:1102, 1:39] 0 0 0 0 0 0 0 1 1 1 ...
 $ RD.post   : num [1:1102, 1:13] 0 0 0 0 0 1 1 1 1 1 ...
 $ g.truncate: int 13
> getcut.fun(sort(ret$haplo$hap.prob, decreasing = TRUE),
>            nn = my.snp$nseq, plot = 1)
\end{Code}

The \code{getcut.fun()} also illustrates a plot based on the information
criterion to decide the truncated dimension, and
the Firgure~\ref{fig:getcut} shows the results,
13 haplotypes will be used in the \code{haplo.post.prob()} function.
\begin{figure}[h]
\begin{center}
\includegraphics[width=6.0in]{./phyclust-graph/getcut}
\caption{A getcut plot for the Crohn's disease dataset.}
\label{fig:getcut}
\end{center}
\end{figure}
