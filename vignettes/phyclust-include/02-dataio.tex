\section[Sequence Data Input and Output]{Sequence Data Input and Output}
\label{sec:dataio}
\addcontentsline{toc}{section}{\thesection. Sequence Data Input and Output}

Two types of sequences are supported in \pkg{phyclust},
nucleotide and SNP. The supported types are stored in \code{.code.type}:
\begin{Code}
> .code.type
[1] "NUCLEOTIDE" "SNP" 
\end{Code}
\pkg{Phyclust} accepts three types of input:
\begin{enumerate}
\item
Data read from a text file in PHYLIP format (Section~\ref{sec:phylip}).
\item
Data read from a text file in FASTA format (Section~\ref{sec:fasta}).
\item
Data simulated by the \code{ms+seqgen} approach (Section~\ref{sec:msseqgen}).
\end{enumerate}
The data reading functions \code{read.*()} will return a list object of 
class {\color{red} \code{seq.data}} (Section~\ref{sec:phylip}).
Suppose we call the returned list object \code{ret}.
Then, \code{ret$org.code} and \code{ret$org} are two matrices that
store the data.
Matrix \code{ret$org.code} contains the original data, e.g. A,G,C,T for
nucleotide, and
\code{ret$org} contains the data formatted for the computer, e.g. 0,1,2,3 for nucleotide.
Matrix \code{ret$org} is translated from \code{ret$org.code}
according to the standard encoding (Section~\ref{sec:coding}) of the chosen data type and most calculations are done with \code{ret$org}.

\pkg{Phyclust} outputs sequence data in two formats:
PHYLIP or FASTA (Section~\ref{sec:save}).





\subsection[Standard coding]{Standard coding}
\label{sec:coding}
\addcontentsline{toc}{subsection}{\thesubsection. Standard coding}

Genetic data are represented internally using an integer code, and only the integer values get passed to the \proglang{C} core.
The two data frames, \code{.nucleotide} and \code{.snp},
are used to map between internal integer code (\code{nid} or \code{sid}) and the human interpretable code (\code{code} or \code{code.l}).
\begin{Code}
> .nucleotide
  nid code code.l
1   0    A      a
2   1    G      g
3   2    C      c
4   3    T      t
5   4    -      -
> .snp
  sid code
1   0    1
2   1    2
3   2    -
\end{Code}
Note that we use ``\code{-}'' to indicate gaps and other non general syntax.
The methods and functions to deal with gaps are still under development.


\subsection[PHYLIP format]{Input PHYLIP format}
\label{sec:phylip}
\addcontentsline{toc}{subsection}{\thesubsection. PHYLIP format}

Some virus data collected from an EIAV-infected pony, \#524~\citep{Baccam2003},
named ``Great pony 524 EIAV rev dataset'',
are provided as an example of PHYLIP-formatted sequence data.
You can view the file with commands
\begin{Code}
> data.path <- paste(.libPaths()[1], "/phyclust/data/pony524.phy", sep = "")
> edit(file = data.path)
\end{Code}
Below, we show the first 5 sequences and first 50 sites. The first line
indicates there are 146 sequences and 405 sites in this file. The sequences
are visible starting from the second line, where the first 10 characters are reserved for
the sequence name or id.
\begin{verbatim}
 146 405
AF314258     gatcctcagg gccctctgga aagtgaccag tggtgcaggg tcctccggca
AF314259     gatcctcagg gccctctgga aagtgaccag tggtgcaggg tcctccggca
AF314260     gatcctcagg gccctctgga aagtgaccag tggtgcaggg tcctccggca
AF314261     gatcctcagg gccctctgga aagtgaccag tggtgcaggg tcctccggca
AF314262     gatcctcagg gccctctgga aagtgaccag tggtgcaggg tcctccggca
\end{verbatim}

By default, function \code{read.phylip()} will read in a PHYLIP file and
assume the file contains nucleotide sequences. It will read in sequences,
translate them using the encoding (Section~\ref{sec:coding}), and
store them in a list object of class \code{seq.data}.
The following example reads the Pony 524 dataset.
\begin{Code}
> data.path <- paste(.libPaths()[1], "/phyclust/data/pony524.phy", sep = "")
> (my.pony.524 <- read.phylip(data.path))
code.type: NUCLEOTIDE, n.seq: 146, seq.len: 405.
> str(my.pony.524)
List of 7
 $ code.type: chr "NUCLEOTIDE"
 $ info     : chr " 146 405"
 $ nseq     : num 146
 $ seqlen   : num 405
 $ seqname  : Named chr [1:146] "AF314258" "AF314259" "AF314260" "AF314261" ...
  ..- attr(*, "names")= chr [1:146] "1" "2" "3" "4" ...
 $ org.code : chr [1:146, 1:405] "g" "g" "g" "g" ...
 $ org      : num [1:146, 1:405] 1 1 1 1 1 1 1 1 1 1 ...
 - attr(*, "class")= chr "seq.data"
\end{Code}

The sample PHYLIP-formatted SNP dataset from a study of Crohn's disease~\citep{Hugot2001} can be loaded with the commands
\begin{Code}
> data.path <- paste(.libPaths()[1], "/phyclust/data/crohn.phy", sep = "")
> (my.snp <- read.phylip(data.path, code.type = .code.type[2]))
code.type: SNP, n.seq: 1102, seq.len: 8.
\end{Code}
Notice, the \code{code.type} argument must specify the data is of type SNP.




\subsection[Input FASTA format]{FASTA format}
\label{sec:fasta}
\addcontentsline{toc}{subsection}{\thesubsection. FASTA format}

The sequence data from another pony, \#625~\citep{Baccam2003},
named ``Great pony 625 EIAV rev dataset'',
is provided in FASTA format.
Here is full-length first sequence in that file. It starts with ``>''
followed by a sequence id and description on the same line.
Subsequent lines contain the actualy sequence until the next line starting with ``>''.
\begin{verbatim}
>AF512608 Equine infectious anemia virus isolate R93.3/E98.1 gp45 and rev
GATCCTCAGGGCCCTCTGGAAAGTGACCAGTGGTGCAGGGTCCTTCGGCAGTCACTACCT
GAAGAAAAAATTCCATCGCAAACATGCATCGCGAGAAGACACCTGGGACCAGGCCCAACA
CAACATACACCTAGCAGGCGTGACCGGTGGATCAGGGAACAAATACTACAGGCAGAAGTA
CTCCAGGAACGACTGGAATGGAGAATCAGAGGAGTACAACAGGCGGCCAAAGAGCTGGAT
GAAGTCAATCGAGGCATTTGGAGAGAGCTACATTTCCGAGAAGACCAAAAGGGAGATTTC
TCAGCCTGGGGCGGTTATCAACGAGCACAAGAACGGCACTGGGGGGAACAATCCTCACCA
AGGGTCCTTAGACCTGGAGATTCGAAGCGAAGGAGGAAACATTTAT
>AF512609 Equine infectious anemia virus isolate R93.2/E105 ...
\end{verbatim}

By default, function \code{read.fasta()} will read in a FASTA file and
assume the file contains nucleotide sequences. 
It also returns a list object of class \code{seq.data}.
The following code example reads the pony \#625 dataset.
\begin{Code}
> data.path <- paste(.libPaths()[1], "/phyclust/data/pony625.fas", sep = "")
> (my.pony.625 <- read.fasta.nucleotide(data.path))
code.type: NUCLEOTIDE, n.seq: 62, seq.len: 406.
> str(my.pony.625)
List of 6
 $ code.type: chr "NUCLEOTIDE"
 $ nseq     : num 62
 $ seqlen   : int 406
 $ seqname  : chr [1:62] "AF512608" "AF512609" "AF512610" "AF512611" ...
 $ org.code : chr [1:62, 1:406] "G" "G" "G" "G" ...
 $ org      : num [1:62, 1:406] 1 1 1 1 1 1 1 1 1 1 ...
 - attr(*, "class")= chr "seq.data"
\end{Code}




\subsection[Saving sequences]{Saving sequences}
\label{sec:save}
\addcontentsline{toc}{subsection}{\thesubsection. Saving sequences \vspace{-0.3cm}}

To save sequences in a file, you can use the functions \code{write.*()}, which are
analogous to the functions \code{read.*()} but take a data matrix \code{X} and
a file name \code{filename}.
With the following code, we save the two pony datasets in PHYLIP and FASTA formats to
the working directory.
\begin{Code}
> # PHYLIp
> write.phylip(my.pony.625$org, "new.625.txt")
> edit(file = "new.625.txt")
> # FASTA
> write.fasta(my.pony.524$org, "new.524.txt")
> edit(file = "new.524.txt")
\end{Code}
